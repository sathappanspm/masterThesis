\documentclass{beamer}
\usetheme{Antibes}
%\usepackage{caption}
%\DeclareCaptionType{copyrightbox}
\graphicspath{{../figures/}}
% \usepackage{subcaption}
% \captionsetup{compatibility=false}
\usepackage{subfig}
\usepackage{graphicx}
\usepackage{multirow}
\usepackage{amsmath, amsfonts, amssymb}
\usepackage[utf8]{inputenc}
\inputencoding{utf8}

\newcommand{\then}{\Rightarrow}
\newcommand{\softor}{\operatornamewithlimits{\tilde{\vee}}}
\newcommand{\softand}{\operatornamewithlimits{\tilde{\wedge}}}
\newcommand{\softthen}{\operatornamewithlimits{\tilde{\then}}}
\newcommand{\softneg}{\operatornamewithlimits{\tilde{\neg}}}


\begin{document}

\title{Forecasting Protests by Detecting Future Time Mentions in News and Social Media
}
\author{Sathappan Muthiah}
\institute{Virginia Tech}
\date{July 2nd, 2014}
\subject{Computer Science}


\frame{\titlepage}



%Table of Contents slide
\begin{frame}
\frametitle{Table of Contents}
\tableofcontents[currentsection]
\end{frame}


\section{Problem Overview}
\begin{frame}
    \frametitle{Problem Overview}
    \begin{itemize}
        \item
            Detecting Future time mentions in relevant media to build a protest forecasting system.
        \item
            Investigate the selective superiorities of Different Social Media.
    \end{itemize}
\end{frame}

\section{motivation}
\begin{frame}
    \frametitle{Motivation}
    \begin{itemize}
        \item
            Around 75\% of the protests are planned, organized, or announced in advance.
        \item
            Identifying these planned protests is an easy way to forecast protests.
    \end{itemize}
\end{frame}


\begin{frame}
    \frametitle{Key Contributions}
    \begin{itemize}
        \item
            Real-Time Prospective Study-most studies until now have been retrospective.
        \item
            Semi-Automatic approach for Learning Keyphrase filters.
        \item
            Handling Mutliple Sources.
        \item
            Reasoning about locations.
        \item
            Handling relative dates - some recent work use only absolute dates.
    \end{itemize}
\end{frame}


\begin{frame}
\frametitle{Overall Framework}
\begin{figure}
    \centering
    \includegraphics[height=0.6\textheight,width=\textwidth]{pipeline}
\end{figure}
\end{frame}

\begin{frame}
    \frametitle{Data Sources}
    \begin{itemize}
        \item
            Long Text
            \begin{itemize}
                \item
                    RSS Feeds
                    \begin{itemize}
                        \item
                            News
                        \item
                            Blogs
                    \end{itemize}
                \item
                    Twitter-URL
            \end{itemize}
        \item
            Short Text
                \begin{itemize}
                    \item
                        Twitter
                \end{itemize}
        \item
            Facebook-Event

    \end{itemize}
\end{frame}

\begin{frame}
    \frametitle{Long Text - RSS Feeds}
    \begin{itemize}
        \item
            Data Duration: November 2012 to March 2014
        \item
            6540 News
        \item
            6540 Blogs
        \item
            Talkwalker Alerts or Google Alerts
    \end{itemize}
\end{frame}

\begin{frame}
    \frametitle{Example - RSS Feed}
\end{frame}

\begin{frame}
    \frametitle{Short Text - Twitter}
    \begin{itemize}
        \item
            Datasift Firehose
        \item
            Duration November 2012 to March 2014
    \end{itemize}
\end{frame}

\begin{frame}
    \frametitle{Example}
\end{frame}

\begin{frame}
    \frametitle{Facebook}
    \begin{itemize}
        \item
            Facebook Graph API
        \item
            Facebook Query Language
    \end{itemize}
\end{frame}

\begin{frame}
    \frametitle{Example}
\end{frame}

\section{Preliminaries}
\begin{frame}
\frametitle{Preliminaries-Probabilistic Soft Logic}
    \begin{itemize}
        \item
            Framework for collective probabilistic reasoning
        \item
            User defined predicates and rules
        \item
            MPE Inference
    \end{itemize}
\end{frame}

\section{Linguistic Preprocessing}
\begin{frame}
    \frametitle{Natural Language Enrichment}
    \begin{itemize}
        \item
            Tokenization
        \item
            Lemmatization
        \item
            Noun Phrase Extraction
        \item
            Named Entity Extraction and Classification
    \end{itemize}
\end{frame}

\begin{frame}
\frametitle{TIMEN Enrichment}
    \begin{itemize}
        \item
            Extraction of Absolute Dates from text
        \item
            Identification of Relative dates like `yesterday, next wednesday' etc.
    \end{itemize}
\end{frame}

\section{Geocoding}
\begin{frame}
\frametitle{Geocoding- RSS Feeds}
\begin{figure}
    \centering
    \includegraphics[width=\textwidth]{psl_pipeline2}
\end{figure}
\end{frame}

\begin{frame}
\frametitle{Geocoding- RSS Feeds Contd.}
    \begin{itemize}
        \item Primary rules
        \begin{flalign*}
            ENTITY&(L, location) \softand REFERSTO(L, locID) &\\
                                &\rightarrow PSLLOCATION(Article, locID) &
        \end{flalign*}


        \begin{flalign*}
            ENTITY&(C, location) \softand IsCountry(C) &\\
                                &\rightarrow ArticleCountry(Article, C) &
        \end{flalign*}


        \begin{flalign*}
            ENTITY&(S, location) \softand IsState(S)&\\
                                    &\rightarrow ArticleCountry(Article, S)&
        \end{flalign*}

    \end{itemize}
\end{frame}

\begin{frame}
\frametitle{Geocoding- RSS Feeds Contd.}
    \begin{itemize}
        \item Secondary rules
    \begin{flalign*}
        ENTITY&(O, organization) \softand REFERSTO(O, locID)&\\
                                &\rightarrow PSLLOCATION(Article, locID) &
    \end{flalign*}


    \begin{flalign*}
        ENTITY&(O, organization) \softand IsCountry(O)&\\
            &\rightarrow ArticleCountry(Article, O)&
    \end{flalign*}


    \begin{flalign*}
        ENTITY&(O, organization) \softand IsState(O)&\\
              &\rightarrow ArticleCountry(Article, O) &
    \end{flalign*}

    \end{itemize}
\end{frame}




\begin{frame}
\frametitle{Geocoding- Twitter}
    \begin{itemize}
        \item
            Geotag of the tweet
        \item
            Twitter ``places" metadata
        \item
            Other text fields (user profile, tweet text)
    \end{itemize}
\end{frame}

\begin{frame}
\frametitle{Geocoding- Twitter}
    \begin{itemize}
        \item
            Facebook Location Objects
        \item
            Facebook Event Venue/location
    \end{itemize}
\end{frame}

\section{Phrase Filtering}
\begin{frame}
\frametitle{Phrase List Development}
    \begin{itemize}
        \item
            Semi-Automatic
        \item
            Different Lists are built for different Sources
        \item
            Seed phrases are identified from analysis of known planned events from print media.
    \end{itemize}

\end{frame}

\begin{frame}
    \frametitle{Dependency Parsing}
    \begin{figure}
        \includegraphics[width=\textwidth]{phraseLearning}
    \end{figure}
\end{frame}

\begin{frame}
    \frametitle{Phrase List for Long Text}
    Example of phrases used for Long Text
\end{frame}


\begin{frame}
    \frametitle{Phrase List for Short text}
    Example of Phrases used for Short text
\end{frame}


\begin{frame}
    \frametitle{System Framework Once again}
    \begin{figure}
        \centering
        \includegraphics[height=0.6\textheight,width=\textwidth]{pipeline}
    \end{figure}
\end{frame}

\section{Evaluation}

\begin{frame}
\frametitle{Evaluation Methodology}
    \begin{itemize}
        \item
            Bipartite Matching
    \end{itemize}
    \begin{figure}
        \includegraphics[height=0.7\textheight]{matching}
    \end{figure}

\end{frame}


\begin{frame}
    \frametitle{Evaluation Methodology Contd}
    \begin{itemize}
        \item
            Date Score
            \begin{equation}
                \operatorname{LS}=1 - \frac{\min(\textrm{distance offset}, 300)}{300}
            \end{equation}
        \item
            Location Score
            \begin{equation}
                \operatorname{DS}=1 - \frac{\min(\textrm{date offset}, \operatorname{INTERVAL})}{\operatorname{INTERVAL}}
            \end{equation}

        \item
            Total Quality Score
            \begin{equation}
                \operatorname{QS}=(\operatorname{DS} + \operatorname{QS})*2
            \end{equation}


    \end{itemize}

\end{frame}

\begin{frame}
    \frametitle{Warnings vs GSR}
    \begin{figure}%
    \centering
    \subfloat{\includegraphics[width=0.45\textwidth]{gsr_distribution}}%
    \qquad
    \subfloat{\includegraphics[width=0.45\textwidth]{pp_dist}}%
    \end{figure}
\end{frame}

\begin{frame}
     \frametitle{Venezuelan Spring}
     \begin{figure}
        \centering
        \includegraphics[scale=0.4]{venezuela}
     \end{figure}
\end{frame}

\begin{frame}
    \frametitle{Venezuelan Violent Protests}
     \begin{figure}
        \centering
        \includegraphics[scale=0.4]{venezuela_violent}
     \end{figure}
\end{frame}

\begin{frame}
    \frametitle{Brazilian Spring}
    \begin{figure}
        \centering
        \includegraphics[scale=0.4]{brazil_june}
    \end{figure}
\end{frame}

\begin{frame}
    \frametitle{Individual Source Level Perfomance}
\begin{table}[tb!]
    \tiny
    \centering
    \begin{tabular}{|*{17}{c|}}
        \hline
        & \multicolumn{4}{ |c| }{News/Blogs} & \multicolumn{4}{ |c| }{Twitter} & \multicolumn{4}{ |c| }{Facebook} & \multicolumn{4}{ |c| }{Combined}\\
        \hline
         & QS & Pr. & Rec. &LT & QS & Pr. & Rec. & LT & QS & Pr. & Rec. & LT & QS & Pr. & Rec. & LT\\
        \hline
        AR &3.14&0.32&0.69&3.94&3.52&{\bf0.78}&0.14&3.14&{\bf3.70}&0.50&0.04&3.00&3.02&0.36&{\bf0.80}&{\bf4.50}\\
        BR &3.14&0.48&0.54&{\bf5.85}&-&-&-&-&{\bf3.62}&{\bf0.76}&0.18&2.46&3.28&0.49&{\bf0.65}&5.15\\
        CL &3.06&0.91&0.67&5.40&{\bf3.52}&{\bf1.00}&0.23&4.29&-&-&-&-&3.16&0.83&{\bf0.80}&{\bf5.92}\\
        CO &2.74&0.90&0.56&{\bf7.44}&3.30&{\bf1.00}&0.15&2.43&{\bf4.00}&{\bf1.00}&0.02&2.00&2.88&0.84&{\bf0.65}&6.47\\
        EC &-&-&-&-&{\bf2.32}&{\bf1.00}&{\bf0.06}&{\bf17.00}&-&-&-&-&{\bf2.32}&{\bf0.50}&{\bf0.06}&{\bf17.00}\\
        MX &2.96&0.88&0.25&{\bf3.69}&3.14&{\bf1.00}&0.02&1.43&{\bf3.72}&0.67&0.01&2.00&3.00&0.87&{\bf0.27}&3.51\\
        SV &{\bf3.22}&{\bf1.00}&{\bf0.03}&{\bf1.0}&-&-&-&-&-&-&-&-&{\bf3.22}&{\bf1.0}&{\bf0.03}&{\bf1.0}\\
        PY &3.38&{\bf1.00}&{\bf0.16}&9.11&3.84&{\bf1.00}&0.04&{\bf11.40}&3.96&{\bf1.00}&0.01&2.00&3.60&0.96&{\bf0.20}&9.35\\
        UY &{\bf3.24}&{\bf1.00}&{\bf0.29}&{\bf2.40}&-&-&-&-&-&-&-&-&3.24&{\bf1.00}&{\bf0.29}&3.24\\
        VE &{\bf3.80}&{\bf1.00}&0.36&{\bf3.27}&3.68&0.97&0.33&2.39&-&-&-&-&3.64&0.99&{\bf0.69}&2.88\\
        ALL &3.34&0.69&0.35&{\bf4.57}&3.62&{\bf0.97}&0.15&2.82&3.66&0.74&0.03&2.44&3.36&0.73&{\bf0.51}&4.08\\
        \hline
    \end{tabular}
\end{table}
\end{frame}

\begin{frame}
    \frametitle{RSS Feeds + Twitter-Urls}
\end{frame}

\begin{frame}
    \frametitle{Twitter}
\end{frame}

\begin{frame}
    \frametitle{Facebook}
\end{frame}

\begin{frame}
    \frametitle{all Sources}
\end{frame}

\begin{frame}
    \frametitle{Performance over time}
    \begin{figure}
        \includegraphics[scale=0.4]{monthlyqs}
    \end{figure}
\end{frame}

\begin{frame}
    \frametitle{Quality Score vs Matching window size}
    \begin{figure}
        \includegraphics[scale=0.4]{matchingwindow}
    \end{figure}
\end{frame}

\begin{frame}
    \frametitle{Lead-Time vs Quality}
    \begin{figure}
        \includegraphics[scale=0.4]{leadTimeVsQS}
    \end{figure}
\end{frame}

\begin{frame}
    \frametitle{Quality Score Distribution}
    \centering
    \begin{figure}
        \includegraphics[scale=0.6]{doubleHump}
    \end{figure}
\end{frame}

\end{document}
